% 杂项宏包
\usepackage[colorlinks,linkcolor=blue]{hyperref}  % 超链接
\usepackage{amsmath}
\usepackage{amssymb}
\usepackage{multicol}
\usepackage{indentfirst}
\usepackage{algorithm}   % 代码块
\usepackage{algorithmicx}   % 代码块
\usepackage{algpseudocode}   % 代码块 
\usepackage{graphicx}  % 图片
\usepackage{subfigure}  % 并排图像
\usepackage{fontspec}
%\usepackage{parskip}
\usepackage[]{caption2}  % 去掉图表的冒号
\renewcommand{\captionlabeldelim}{}  % 去掉图表的冒号
\usepackage{booktabs} % 三线表
\usepackage{pdfpages}  % 直接插入pdf页面
% \setlength{\parindent}{2em}

% 字体设置
\usepackage{xeCJK}
%%%%%%%%%%% CJK下设置中文字体 %%%%%%%%%%%%%
\setCJKmainfont{SimSun.ttf}  % 中文字体
\setCJKsansfont{simhei.ttf}
\setCJKmonofont{simkai.ttf}
\setCJKfamilyfont{song}{SimSun.ttf}
\newcommand{\song}{\CJKfamily{song}}
\setCJKfamilyfont{kai}{simkai.ttf}
\newcommand{\kai}{\CJKfamily{kai}}
\setCJKfamilyfont{fang}{simfang.ttf}
\newcommand{\fang}{\CJKfamily{fang}}
\setCJKfamilyfont{hei}{simhei.ttf}
\newcommand{\hei}{\CJKfamily{hei}}
\setCJKfamilyfont{li}{SIMLI.TTF}
\newcommand{\li}{\CJKfamily{li}}
\setmainfont[BoldFont=Times-Bold.ttf, ItalicFont=Times-Italic.otf]{Times.ttf}  % 英文字体
\setsansfont{Source Sans Pro}
\setmonofont{simkai.ttf}



\usepackage{gbt7714}  % 国标参考文献
\renewcommand\refname{\hei \sihao 参考文献}

% 页面布局设置
\usepackage{geometry}
\geometry{a4paper, scale=0.8}  % 页面边距
% \setlength{\parskip}{1.15em}  % 段落之间空 0.5 行


% 其他设置
%%%%%%%%%%%  设置字体大小 %%%%%%%%%%%%%
\newcommand{\chuhao}{\fontsize{42pt}{\baselineskip}\selectfont}   %%%初号字体
\newcommand{\xiaochuhao}{\fontsize{36pt}{\baselineskip}\selectfont}    %%%小初号字体
\newcommand{\yihao}{\fontsize{28pt}{\baselineskip}\selectfont}   %%%一号字体,以此类推
\newcommand{\erhao}{\fontsize{21pt}{\baselineskip}\selectfont}
\newcommand{\xiaoerhao}{\fontsize{18pt}{\baselineskip}\selectfont}
\newcommand{\sanhao}{\fontsize{15.75pt}{\baselineskip}\selectfont}
\newcommand{\sihao}{\fontsize{14pt}{\baselineskip}\selectfont}
\newcommand{\xiaosihao}{\fontsize{12pt}{\baselineskip}\selectfont}
\newcommand{\wuhao}{\fontsize{10.5pt}{\baselineskip}\selectfont}
\newcommand{\xiaowuhao}{\fontsize{9pt}{\baselineskip}\selectfont}
\newcommand{\liuhao}{\fontsize{7.875pt}{\baselineskip}\selectfont}
\newcommand{\qihao}{\fontsize{5.25pt}{\baselineskip}\selectfont}

%%%% 设置 section 属性 %%%%
\makeatletter
\renewcommand\section{\@startsection{section}{1}{\z@}%
  {-1.5ex \@plus -.5ex \@minus -.2ex}%
  {.5ex \@plus .1ex}%
  {\normalfont\sihao\fang}}  %%这边要是不想要宋体三号字,可以把song改成之前定义的字体和字号。
%{\normalfont\sihao\CJKfamily{fs}}}  %%比如改成仿宋四号,前面定义好了,后面只需要改几个字母就可以了。
\makeatother
%%%% 设置 subsection 属性 %%%%
\makeatletter
\renewcommand\subsection{\@startsection{subsection}{1}{\z@}%
	{-1.25ex \@plus -.5ex \@minus -.2ex}%
	{.4ex \@plus .1ex}%
	{\normalfont\wuhao\hei}}
\makeatother
%%%% 设置 subsubsection 属性 %%%%
\makeatletter
\renewcommand\subsubsection{\@startsection{subsubsection}{1}{\z@}%
	{-1ex \@plus -.5ex \@minus -.2ex}%
	{.3ex \@plus .1ex}%
	{\normalfont\wuhao\song}}
\makeatother
% 章节从 0 开始
\setcounter{section}{-1}

% 算法标题设置
\floatname{algorithm}{\xiaowuhao 算法}
\renewcommand{\algorithmicrequire}{\textbf{输入:}}
\renewcommand{\algorithmicensure}{\textbf{输出:}}

\renewcommand{\figurename}{\xiaowuhao \hei 图}
\renewcommand{\tablename}{\xiaowuhao \hei 表}

