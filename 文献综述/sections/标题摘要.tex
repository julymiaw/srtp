
\begin{center}
  \erhao \sffamily 基于机器学习的源代码错误定位研究报告

  \vspace{0.3cm}

  \xiaosihao \ttfamily 吴清晏$1$ \quad 杨锦波$2$ \quad 张闻启$3$ \quad 董子翔$4$ \quad 万奕含$5$

\end{center}

\xiaowuhao{
  \noindent \sffamily 摘要: \normalfont 在本文中,我们探讨了基于机器学习的错误定位方法,包括基于向量空间的方法、基于学习排名的方法、基于代码修改历史的方法、基于执行信息的方法,以及基于深度学习的方法。近年来深度学习模型在自然语言处理领域取得了重大突破,这将会催生出更多新的高效的静态错误定位方法。希望通过本文的介绍,能够帮助读者理解机器学习在源代码缺陷定位中的应用,为后续的研究工作提供参考。

  \noindent \sffamily 关键词:\normalfont 错误定位;错误报告;深度学习;预训练模型
}

\begin{center}
  \sihao Fault-localization based on machine learning models

  \vspace{0.3cm}

  \xiaosihao \ttfamily Qingyan Wu$^1$ \quad Jinbo Yang$^2$ \quad Wenqi Zhang$^3$ \quad Zixiang Dong$^4$ \quad Yihan Wan$^5$

  \xiaowuhao (1. Southeast University, 211189, Nanjing, China)

\end{center}
\xiaowuhao{
  \noindent \textbf{Abstract:} In this paper, we explore various fault localization methods based on machine learning, including those based on vector space models, learning to rank, code modification history, execution information, and deep learning. In recent years, deep learning models have achieved significant breakthroughs in the field of natural language processing, which will give rise to more efficient static error localization methods. Through this paper, we hope to help readers understand the application of machine learning in source code defect localization and provide a reference for subsequent research work.

  \noindent \textbf{Keywords: } fault localization; bug report; deep learning; pre-trained model
}

